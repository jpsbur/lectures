\documentclass[12pt]{beamer}

\usepackage[russian,english]{babel}
\usepackage[utf8]{inputenc}
\usepackage[T2A]{fontenc}
\usepackage{beamerthemeshadow}
\usepackage{color}

\definecolor{orange}{rgb}{1,0.5,0}
\definecolor{darkgreen}{rgb}{0,0.8,0}
\definecolor{darkblue}{rgb}{0,0,0.8}

\begin{document}
\title{Базы данных для хранения списков, или Зачем нужны декартовы деревья, чтобы поставить сердечко}
\author{Юрий Петров}
\date{июль 2014 года}

\frame{\titlepage}

\frame{
  \frametitle{Задача}

  Задача
  \pause
  \begin{itemize}
    \item Поддерживать списки целых чисел
    \pause
    \item Уметь:
      \begin{itemize}
        \item добавлять элемент
        \item удалять элемент
        \item проверять наличие элемента
        \item получить список или его часть
      \end{itemize}
    \pause
    \item Пример --- ``лайки'':
      \begin{itemize}
        \item Список --- запись
        \item Элемент списка --- тот, кто поставил ``лайк''
      \end{itemize}
  \end{itemize}
}

\frame{
  \frametitle{Статический подход}

  Статический подход
  \pause
  \begin{itemize}
    \item Предположим: запросы на изменение уже выполнены
    \pause
    \item Выберем структуру
      \begin{itemize}
        \item Упорядоченный массив для каждого списка
        \item Просто построить, просто получить ответ
      \end{itemize}
  \end{itemize}
}

% Задача
% Подход статический
%  отсортированный список
%  хеш-таблица?
% Подход динамический
%  деревья поиска
%  хеш-таблицы?
% Подход комбинированный
% Переиндексирование

\end{document}
