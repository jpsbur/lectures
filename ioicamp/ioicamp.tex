\documentclass[12pt]{beamer}

\usepackage[russian,english]{babel}
\usepackage[utf8]{inputenc}
\usepackage[T2A]{fontenc}
\usepackage{beamerthemeshadow}
\usepackage{color}

\definecolor{orange}{rgb}{1,0.5,0}
\definecolor{darkgreen}{rgb}{0,0.8,0}
\definecolor{darkblue}{rgb}{0,0,0.8}

\begin{document}
\title{Сборы для подготовки к IOI}
\author{Юрий Петров}
\date{7 декабря 2013 года} 

\frame{\titlepage}

\frame{
  \frametitle{О чём речь}

  \pause
  \begin{itemize}
    \item IOI --- Международная олимпиада по информатике
    \pause
    \item Проводится ежегодно с 1989 года
    \pause
    \item Индивидуальный зачёт
    \pause
    \item Страна отправляет четырёх участников
  \end{itemize}
}

\frame{
  \frametitle{Замечание}

  В силу субъективности восприяти и не только
  \pause
  \begin{itemize}
    \item ``Сейчас'' --- 2010 год и далее
    \item ``Раньше'' --- 2009 год и ранее
    \item ``Совсем давно'' --- до 2000 года
  \end{itemize}
}

\frame{
  \frametitle{Цели сборов}

  \pause
  \begin{itemize}
    \item На IOI нужна команда, значит, её нужно найти
    \pause
    \item На самом деле, команда нужна будет ещё и через год
    \pause
    \item Воспользуемся матиндукцией...
    \pause
    \item Но выстраивать систему образования от IOI не стоит, не правда ли?
  \end{itemize}
}

\frame{
  \frametitle{Итак, настоящие цели}

  \pause
  \begin{itemize}
    \item Предположим, система образования и Всероссийская олимпиада уже есть
    \pause
    \item Тогда действительно, нужны:
      \begin{itemize}
        \item Команда на ближайшую олимпиаду
        \item Ближайший резерв
      \end{itemize}
  \end{itemize}
}

\frame{
  \frametitle{Следствие --- состав}

  \pause
  \begin{itemize}
    \item Потециальные участники команды
    \pause
      \begin{itemize}
        \item Чем дальше, тем реже туда можно попасть сразу
        \item Одиннадцатиклассники могут оказаться только здесь
        \item Пять-семь человек
      \end{itemize}
    \pause
    \item Потенциальные участники команды на следующие годы
    \pause
      \begin{itemize}
        \item 10 класс и младше
        \item Должны быть призёрами Всероссийской олимпиады
        \item Почти всегда: все возможные призёры 9 и младше
        \item Некоторые успешные десятиклассники
      \end{itemize}
    \pause
    \item Можно ли добавить тех, кто почти подошёл?
  \end{itemize}
}

\frame{
  \frametitle{Попадание на сборы}

  \pause
  \begin{itemize}
    \item Основной критерий --- Всероссийская олимпиада:
      \begin{itemize}
        \item Охват
        \item Ясность результатов
      \end{itemize}
    \pause
    \item Внесённый в последнее время формализм
    \pause
    \item Прошлые заслуги
    \pause
    \item Исключения?
  \end{itemize}
}

\frame{
  \frametitle{Две части сборов}

  \pause
  \begin{itemize}
    \item Лето
      \begin{itemize}
        \item Отбор
        \item Подготовка
      \end{itemize}
    \pause
    \item Зима
      \begin{itemize}
        \item Подготовка к лету
        \item Дополнение резерва --- можно взять тех, кто ``не поместился'' летом
      \end{itemize}
  \end{itemize}
}

\frame{
  \frametitle{Третья часть сборов --- дистанционные туры}

  \pause
  \begin{itemize}
    \item Нововведение --- с 2012 года
    \pause
    \item Проводятся в течение года
    \pause
    \item Поддерживают потенциальную команду в форме
  \end{itemize}
}

\frame{
  \frametitle{Структура сборов}

  \pause
  \begin{itemize}
    \item 10--15 дней
    \pause
    \item Туры ``в реальных условиях''
    \pause
    \item Разборы задач
    \pause
    \item Теоретические семинары
    \pause
    \item Практические семинары
    \pause
    \item Знакомство с задачами прошедших IOI
    \pause
    \item Лекции
    \pause
    \item Дорешивание --- всё свободное время
    \pause
    \item Открытые соревнования: TC, CF, APIO, USACO
    \pause
    \item Внеучебная деятельность
  \end{itemize}
}

\frame{
  \frametitle{Структура сборов: туры}

  \pause
  \begin{itemize}
    \item 5 часов
    \pause
    \item Самостоятельное решение задач
    \pause
    \item Задачи составлены специально или гарантированно неизвестны участникам
  \end{itemize}
}

\frame{
  \frametitle{Структура сборов: туры по-новому}

  \pause
  \begin{itemize}
    \item Тренировочные туры: как раньше
    \pause
    \item Проверочные туры: по известным задачам IOI
    \pause
    \item Отборочные туры: как раньше, их сумма считается в отбор
  \end{itemize}
}

\frame{
  \frametitle{Структура сборов: лекции}

  \pause
  \begin{itemize}
    \item Раньше
      \pause
      \begin{itemize}
        \item Две-три группы разного уровня
        \pause
        \item Для многих --- единственный способ узнать стандартные темы
      \end{itemize}
    \pause
    \item Сейчас
      \begin{itemize}
        \item Общее количество меньше
        \pause
        \item Темы более экзотичны
        \pause
        \item Причины --- ЛКШ, интернет, кружки
        \pause
        \item Работа над техническими и тактическими моментами
      \end{itemize}
  \end{itemize}
}

\frame{
  \frametitle{Отбор: возможные критерии}

  \pause
  \begin{itemize}
    \item Результаты на Всероссийской олимпиаде
    \pause
    \item Результаты сборов
    \pause
    \item Суммарные и прочие выбранные результаты
    \pause
    \item Психологическая устойчивость
    \pause
    \item Общее впечатление
    \pause
    \item Возраст
  \end{itemize}
}

\frame{
  \frametitle{Отбор: ранее}

  \pause
  \begin{itemize}
    \item Основа --- результаты сборов
    \pause
    \item Фактическое решение за руководителями
    \pause
    \item Исключения:
      \begin{itemize}
        \item Взять школьника, бывшего на IOI и пропустившего сборы на других сборах
        \item Взять школьника, бывшего на IOI и пропустившего сборы по болезни
        \item Взять школьника более младшего класса с более низким результатом
      \end{itemize}
  \end{itemize}
}

\frame{
  \frametitle{Отбор: сейчас}

  Пришёл формализм
  \pause
  \begin{itemize}
    \pause
    \item Есть выделенные отборочные туры --- около четырёх
    \pause
    \item Их результаты суммируются, и являются единственным критерием
    \pause
    \item Решение не изменить
    \pause
    \item Цена ошибки выше
  \end{itemize}
}

\frame{
  \frametitle{Что мы упускаем}

  Из-за формализма и не только
  \begin{itemize}
    \item Школьники, не оказавшиеся успешными во Всероссийской олимпиаде
    \item ``Внезапные'' школьники
  \end{itemize}
}

\frame{
  \frametitle{Кто это делает}

  \begin{itemize}
    \item Участники и победители прошлых IOI
    \pause
    \item Студенты и преподаватели университетов
    \pause
    \item Зачастую одновременно :)
    \pause
    \item Имеется преемственность
    \pause
    \item От трёх до десяти человек
  \end{itemize}
}

\frame{
  \frametitle{Что делается}

  \begin{itemize}
    \item Подготавливается порядка десяти дней по три-четыре задачи
    \pause
    \item Проводится пять--десять лекций
    \pause
    \item Проводятся семинары, разборы задач
    \pause
    \item Работа со школьниками на дорешивании
    \pause
    \item Общение на тактические и психологические темы
  \end{itemize}
}

\frame{
  \frametitle{Как это делается}

  \begin{itemize}
    \item Выбор тем задач и лекций
    \pause
    \item Сбалансированные туры и тематические туры
    \pause
    \item IOI Syllabus
  \end{itemize}
}

\frame{
  \frametitle{Технологии}

  \begin{itemize}
    \item Автоматизированная тестирующая система
    \pause
    \item Сейчас сборы полностью под GNU/Linux
  \end{itemize}
}

\frame{
  \frametitle{Руководство}

  \begin{itemize}
    \item Не менялось с первого участия в IOI
    \pause
    \item Чем дальше, тем больше ``участвует'' в процессе
    \pause
    \item Вносит формализм
    \pause
    \item Сборами можно (было?) заниматься так, чтобы оргвопросы не беспокоили
  \end{itemize}
}

\frame{
  \frametitle{Зачем ехать на сборы}

  \begin{itemize}
    \item Поехать и выиграть IOI
    \pause
    \item Научиться чему-то
      \begin{itemize}
        \item Порог входа довольно высокий
        \pause
        \item Цель хорошая, но эффект может оказаться другим
        \pause
        \item Кружок или ЛКШ могут быть эффективней
        \pause
        \item Учить простым вещам будут, но мало \pause (а стоило бы)
      \end{itemize}
  \end{itemize}
}

\frame{
  \frametitle{Результаты}

  \begin{itemize}
    \item С 1992 года все четыре участника всегда получали медали
    \item С 1993 года хотя бы одна золотая медаль
    \item По четыре золотых медали в 2004 и 2012
    \item 12 бронзовых, 29 серебрянных, 47 золотых медалей --- всего 88
  \end{itemize}
}

\frame{
  \frametitle{Ссылки}

  \begin{itemize}
    \item http://ioinformatcs.org/
    \pause
    \item http://neerc.ifmo.ru/school/
    \pause
    \item http://acm.math.spbu.ru/
    \pause
    \item http://snarknews.info/
  \end{itemize}
}

\end{document}
