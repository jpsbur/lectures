\documentclass[12pt]{beamer}

\usepackage[russian,english]{babel}
\usepackage[utf8]{inputenc}
\usepackage[T2A]{fontenc}
%\usepackage{beamerthemeshadow}
\usepackage{color}

\definecolor{orange}{rgb}{1,0.5,0}
\definecolor{darkgreen}{rgb}{0,0.8,0}
\definecolor{darkblue}{rgb}{0,0,0.8}

\usetheme{Goettingen}

\begin{document}
\title{KDB/Engine: базы данных}
\author{Юрий Петров}
\date{июль 2014}

\frame{\titlepage}

\frame{
  \frametitle{О чём речь}

  \pause
  \begin{itemize}
    \item Нужно хранить большие объёмы данных:
    \begin{itemize}
      \item списки целых чисел
      \item текстовые сообщения
      \item рейтинги с затуханием
      \item базы для поиска
      \item et cetera
    \end{itemize}
    \pause

    \item Для каждой цели --- свой набор структур данных
    \pause

    \item Нужна общая схема построения баз
  \end{itemize}
}

\frame{
  \frametitle{Требования к схеме}

  \pause
  \begin{itemize}
    \item Универсальность
    \pause
    \item Масштабируемость
    \pause
    \item Отказоустойчивость
  \end{itemize}
}

\frame{
  \frametitle{Требования к схеме: универсальность}

  \pause
  \begin{itemize}
    \item Общая вне зависимости от хранимых данных
    \pause
    \item Расширяется и усложняется при необходимости
  \end{itemize}
}

\frame{
  \frametitle{Требования к схеме: масштабируемость}

  \pause
  \begin{itemize}
    \item Поддерживает распределённое хранение
    \pause
    \item Поддерживает изменение масштабов данных
  \end{itemize}
}

\frame{
  \frametitle{Требования к схеме: отказоустойчивость}

  \pause
  \begin{itemize}
    \item Удобство резервного копирования
    \pause
    \item Быстрое восстановление в случае отказа
  \end{itemize}
}

\frame{
  \frametitle{Схема: состояние, лог, снимок}

  \pause
  \begin{itemize}
    \item Состояние базы --- все данные, записанные в неё с начала времён
    \pause
    \item Состояние изменяется внешними запросами
    \pause
    \item Лог --- список всех запросов изменения
    \pause
    \item Снимок --- описание состояния базы на фиксированный момент
  \end{itemize}
}

\frame{
  \frametitle{Схема: лог}

  \pause
  \begin{itemize}
    \item Лог состоит из событий
    \pause
    \item Событие:
      \begin{itemize}
        \item Общий заголовок
        \item Специфичные данные
      \end{itemize}
    \pause
    \item События упорядочены
    \pause
    \item Получить состояние базы --- воспроизвести все события
      до текущего момента
  \end{itemize}
}

\frame{
  \frametitle{Схема: снимок}

  \pause
  \begin{itemize}
    \item Если событий много, проигрывать их долго
    \pause
    \item Можно записывать состояние базы на данный момент
    \pause
    \item Момент --- это позиция в логе
  \end{itemize}
}

\frame{
  \frametitle{Схема: масштабирование}

  \pause
  \begin{itemize}
    \item Выберем целочисленный параметр $x$ дробления
      \begin{itemize}
        \item Идентификатор списка
        \item Идентификатор пользователя
        \item Хеш от ключа
      \end{itemize}
    \pause
    \item Настроим $n$ баз
    \pause
    \item Запишем событие в экземпляр $x \mod n$
    \pause
    \item Основная проблема --- выбор параметра дробления
  \end{itemize}
}

\frame{
  \frametitle{Схема: перемасштабирование}

  \pause
  \begin{itemize}
    \item Смена параметра дробления сложна, но возможна
    \pause
    \item При увеличении нагрузки каждый экземпляр
      можно раздробить в два или более раз
  \end{itemize}
}

\frame{
  \frametitle{Создание снимков}

  \pause
  \begin{itemize}
    \item Online-подход
    \begin{itemize}
      \item Состояние в памяти и снимок очень похожи
      \item Можно записывать снимок как \textit{почти} копию состояния
    \end{itemize}
    \pause
    \item Offline-подход
    \begin{itemize}
      \item Алгоритм создания снимка может быть сложным
      \item Можно воспроизводить события лога отдельно
    \end{itemize}
  \end{itemize}
}

\frame{
  \frametitle{Создание снимков: online}

  \pause
  \begin{itemize}
    \item Плюсы
      \begin{itemize}
        \item Можно переиспользовать уже созданные структуры
        \item Нет дополнительной нагрузки на чтение с диска
      \end{itemize}
    \pause
    \item Минусы
      \begin{itemize}
        \item Требуется запущенный движок
        \item Ресурсы можно выделить только на том же сервере
        \item Сложнее обновить версию
      \end{itemize}
  \end{itemize}
}

\frame{
  \frametitle{Создание снимков: offline}

  \pause
  \begin{itemize}
    \item Плюсы
      \begin{itemize}
        \item Можно создавать снимки в любом месте
        \item Просто сделать новую версию
      \end{itemize}
    \pause
    \item Минусы
      \begin{itemize}
        \item Требуется отдельный процесс, полностью читающий лог
        \item Больше нагрузка на диск
      \end{itemize}
  \end{itemize}
}

\frame{
  \frametitle{Резервное копирование}

  \pause
  \begin{itemize}
    \item Потеря снимка не чревата потерей данных, но влияет на время
    \pause
    \item Резервную копию лога можно просто дописывать в конец
    \pause
    \item Эффективно копировать на другой диск
  \end{itemize}
}

\frame{
  \frametitle{Репликация}

  \pause
  \begin{itemize}
    \item Масштабирование нагрузки: можно запустить несколько
      экземпляров движка на одном и том же логе
    \pause
    \item Резервное копирование: можно онлайн копировать данные
      на другие серверы
    \pause
    \item Перенос данных
  \end{itemize}
}

\frame{
  \frametitle{Проверка целостности}

  \pause
  \begin{itemize}
    \item Циклическая проверка целостности (CRC32)
    \pause
    \item Можно улучшить при наличии схемы
  \end{itemize}
}

\frame{
  \frametitle{Шифрование и сжатие}
  
  \pause
  \begin{itemize}
    \item Логи можно сжимать
      \begin{itemize}
        \item Увеличивается скорость чтения с диска
        \item Незначительно уменьшается скорость обработки
      \end{itemize}
    \pause
    \item Логи можно шифровать
      \begin{itemize}
        \item Важно сначала сжимать, потом шифровать
        \item Важно думать о том, где хранить ключ
      \end{itemize}
    \pause
  \end{itemize}
}

\frame{
  \frametitle{Разделение чтения и записи}

  \pause
  \begin{itemize}
    \item Запустить два экземпляра:
      \begin{itemize}
        \item Один только записывает данные в лог
        \item Второй читает и выполняет запросы
          (работает как читающая реплика)
      \end{itemize}
    \pause
    \item Увеличивает отказоустойчивость
    \pause
    \item Почти не влияет на остальные параметры
  \end{itemize}
}

\frame{
  \frametitle{Заключение}

  \pause
  \begin{itemize}
    \item \texttt{https://github.com/vk-com/kphp-kdb}
  \end{itemize}
}

% Запросы
% Бинлоги
% Снимки
% Запуск
% Переиндексирование
% Резервное копирование
% Репликация
% Устойчивость
% Проверка целостности
% Разделение чтения и записи

% GMS?

\end{document}
